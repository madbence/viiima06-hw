\documentclass[a4paper,12pt]{article}

\usepackage{xltxtra}
\usepackage{graphicx}
\usepackage{setspace}
\setlength{\parindent}{0pt}
\setlength{\parskip}{8pt plus 3pt minus 3pt}
\onehalfspacing
\usepackage{fancyhdr}
\pagestyle{fancy}
\usepackage[magyar]{babel}
\usepackage{caption}
\usepackage{subcaption}
\usepackage{pgfplots}

\lhead{Nagyteljesítményű párhuzamos feldolgozás}
\rhead{Dányi Bence}
\rfoot{\today}

\begin{document}

Az ábrán CPU illetve CUDA futtatás mellett mért futási időket láthatjuk.
A mérések linux operációs rendszeren (\texttt{x86\_64 Linux 4.2.5-1-ARCH}),
Intel Core i5-4690 processzoron (3.9 GHz), és GeForce GTX980 grafikus kártyán
készültek (CUDA 7.5, 2048 CUDA mag), 5.2.0-ás \texttt{g++} és 7.5.17-es \texttt{nvcc} fordítóval.

\begin{figure}[h!]
\centering
\begin{tikzpicture}
\begin{loglogaxis}[
  xlabel={Iterációk [$\times 10$ db]},
  ylabel={Futási idő [s]},
  % xmin=0, xmax=100,
  % ymin=0, ymax=120,
  % xtick={0,20,40,60,80,100},
  % ytick={0,20,40,60,80,100,120},
  legend pos=north west,
  % ymajorgrids=true,
  % grid style=dashed,
]

\addplot[
  color=blue,
  mark=square,
]
coordinates {
(1,0.967)
(2,1.054)
(5,1.330)
(10,1.733)
(20,2.581)
(50,5.176)
(100,9.453)
(200,18.056)
(500,43.863)
};

\addplot[
  color=red,
  mark=square,
]
coordinates {
(1,1.856)
(2,3.637)
(5,8.961)
(10,17.856)
(20,35.745)
(50,89.038)
(100,179.132)
(200,355.909)
(500,890.035)
};
\legend{CUDA,CPU}

\end{loglogaxis}
\end{tikzpicture}
  \caption{A futási idő változása a különböző gyorsítási módok mellett}
\end{figure}

Az iterációk számának növelésével a képminőség javul (az algoritmus kifinomultságának függvényében),
alacsony számú iterációnál a különböző adminisztratív (pl. memória) műveletek még összemérhetőek a fő számítási folyamatokkal, ez a különbség később eltűnik.

\begin{figure}[h!]
  \centering
  \begin{subfigure}[b]{0.19\textwidth}
    \includegraphics[width=\textwidth]{render-no-cuda-2.png}
  \end{subfigure}
  \begin{subfigure}[b]{0.19\textwidth}
    \includegraphics[width=\textwidth]{render-no-cuda-5.png}
  \end{subfigure}
  \begin{subfigure}[b]{0.19\textwidth}
    \includegraphics[width=\textwidth]{render-no-cuda-20.png}
  \end{subfigure}
  \begin{subfigure}[b]{0.19\textwidth}
    \includegraphics[width=\textwidth]{render-no-cuda-50.png}
  \end{subfigure}
  \begin{subfigure}[b]{0.19\textwidth}
    \includegraphics[width=\textwidth]{render-no-cuda-200.png}
  \end{subfigure}
  \caption{Eredmények különböző számú iterációk mellett}
\end{figure}

\begin{figure}[h!]
\centering
\begin{tikzpicture}
\begin{semilogxaxis}[
  xlabel={Iterációk [$\times 10$ db]},
  ylabel={Speedup},
  legend pos=north west,
]

\addplot[
  color=blue,
  mark=square,
]
coordinates {
(1,1.919)
(2,3.449)
(5,6.911)
(10,10.603)
(20,13.643)
(50,17.60)
(100,18.696)
(200,19.698)
(500,20.266)
};

\end{semilogxaxis}
\end{tikzpicture}
  \caption{A futási idő változása a különböző gyorsítási módok mellett},
\end{figure}

\end{document}
